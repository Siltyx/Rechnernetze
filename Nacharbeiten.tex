\documentclass[12pt,a4paper,oneside,german]{article}
\usepackage[utf8]{inputenc} 
\title{Rechnernetze Nacharbeiten}
\author{Sebastian Brückner}
\renewcommand*\contentsname{Inhalt}

%Sparche
\usepackage[german]{babel} 

%Versioning
\usepackage[owncaptions]{vhistory}

%Fußzeile
\usepackage{fancyhdr}
\pagestyle{fancy}
\fancyhf{} %alle Kopf- und Fußzeilenfelder bereinigen
\fancyfoot[R]{\thepage} %Seitennummer
\fancyfoot[C]{v.\vhCurrentVersion~vom \vhCurrentDate} %Seitennummer

\begin{document}
\maketitle

%Aktuelle Version Eintragen
\begin{versionhistory}
	\vhEntry{0.1}{26.03.14}{Sebastian Brückner}{Arten von Rechnernetzen}
\end{versionhistory}

\newpage
\tableofcontents
\section{Arten von Rechnernetzen (Aufgaben)}	
	\subsection{Lastverbund}
		\textbf{Ziel:} Alle verbundenen Rechner sollen nach Möglichkeit voll ausgelastet werden. \\\\
		Die Aufgaben die an der Rechnerverbund gestellt werden, 
		werden gleichmäßig an alle Rechner verteilt, 
		um so eine maximal Ressourcen Auslastung bzw. gleichmäßige Auslastung der verbundenen Rechner zu erreichen. \cite{Lastv} \\
		\textbf{Beispiel:} Das Rechnernetz besteht aus den Computern X und Y. Diese erhalten die Aufgaben A1, A2, A3, A4.
		Diese werden nun auf die Rechner aufgeteilt. So könnte X die Aufgaben A1 und A4 erhalten, während Y A2 und A3 erhält.

	\subsection{Leistungsverbund}
		\textbf{Ziel:} Mehr Leistung zum abarbeiten von Aufgaben \\\\
		Die Aufgaben werden parallelisiert und so an alle Rechner im Netzwerk verteilt (ähnlich wie ein Multiprozessorsystem).
		Rechner mit verschiedenen Funktionalitäten (Rechner mit Vorteilen beim bearbeiten eines Teilproblems) können zu einer virtuellen Universalmaschine verbunden werden. \cite{Leistungsv}\\
		\textbf{Beispiel:} Das Rechnernetz besteht aus den Computern X und Y. Diese erhalten die Aufgaben A.
		A wird nun in die Teilaufgabe AT1 und AT2 geteilt. X bearbeitet Teilproblem AT1 und Y bearbeitet AT2.

	\subsection{Verfügbarkeitsverbund}
		\textbf{Ziel:} Hohe Verfügbarkeit, Betriebssicherheit und geringe Ausfallzeit (Hochverfügbarkeit). \\\\
		Durch Redundanzen (Systemdopplung) werden Rechner so verbunden, das der Ausfall einzelner Rechner/Komponenten nicht den Ausfall des ganzen Systems nach sich zieht. \cite{Verfuegbarkeitsv}\\
		\textbf{Beispiel:} Das Rechnernetz besteht aus den identischen Computern X und Y. Diese erhalten die Aufgaben A.
		Der Rechner X fällt aus. Y kann die Aufgabe A immer noch alleine bearbeiten, da er die gleichen Funktionalitäten bietet wie X.
	
	\subsection{Funktionsverbund}
		\textbf{Ziel:} Bereitstellen spezieller Funktionen im Netzwerk\\\\
		Bestimmte Rechner bieten spezielle Funktionen, die andere nicht haben.
		Damit alle Rechner im Netzwerk diese Funktionen nutzen können, werden sie zu einem Funktionsverbund vernetzt.\cite{Funktionsv}\\
		\textbf{Beispiel:} Das Rechnernetz besteht aus den Computern X und Y. Y bearbeitet die Aufgabe A.
		Zum Lösen der Aufgabe wird eine bestimmte Funktionalität benötigt die Y nicht hat, diese stellt X im Netz bereit.
	\subsection{Datenverbund}
		\textbf{Ziel:} Bereitstellen von Daten \\\\
		Daten können geographisch von einander getrennt sein. (Um z.B. unnötig weite Transporte über ein Netzwerk zu vermeiden).
		Der Datenverbund Koppelt diese voneinander getrennten Daten logisch, sodass keine Dateninkonsistens entstehen kann.\\
		\textbf{Beispiel:} Datenbanken
	\subsection{Kommunikationsverbund/Nachrichtenverbund}
		\textbf{Ziel:} Austausch von Nachrichten \\\\
		Einzelne Teilnehmen können Nachrichten Untereinander austauschen
		
		
		


\begin{thebibliography}{}
	\bibitem{Lastv} http://www.itwissen.info/Lastverbund-load-compound.html
	\bibitem{Leistungsv} http://www.itwissen.info/Leistungsverbund-line-compound.html
	\bibitem{Verfuegbarkeitsv} http://www.itwissen.info/Verfuegbarkeitsverbund-high-availability-distributed-system.html
	\bibitem{Funktionsv}  http://www.itwissen.info/Funktionsverbund-function-sharing.html
	\bibitem{Datenv.} http://www.itwissen.info/Datenverbund-data-interconnection.html
\end{thebibliography}

\end{document}
